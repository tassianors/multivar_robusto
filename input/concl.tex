\section{Conclus�es}
\label{sec:concl}

Foi apresentado neste trabalho uma breve descri��o do funcionamento do
Filtro de Kalman, suas in�meras aplica��es nas mais diversas atividades
e ramos da engenharia moderna. Apresentou-se o diferencial da utilizac�o 
do filtro devido a sua robustez para implementa��o no que diz respeito ao
mundo digital e sua robustez quando existe ruido ou incerteza na planta em
quest�o.

Apresentou-se tamb�m o problema de LQR para tempo discreto, sua modelagem
e a realiza��o do filtro de kalman pelo principio da separa��o, onde a
modelagem do filtro pode ser entendida/resolvida pela resolu��o de um problema
de LQR e outro de LQG.

Apresentou-se tamb�m dois tipos principais para a realiza��o do filtro de kalman:
realiza��o por estima��o (\ref{fig:kalman_bucy_filter}) e a realiza��o em 
cascata (\ref{fig:cascade_realization}).

Com estes pontos apresentados aqui e com a grande aplicabilidade dos filtros
de kalman em problemas comuns da engenharia, observamos que este aparato
tem grande import�ncia e suas aplica��es (\ref{sec:applications}) n�o se 
restringem �s apresentadas neste trabalho.

