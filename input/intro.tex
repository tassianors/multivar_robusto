%===============================================================================
\section{Introdu��o}

Neste trabalho ser� apresentado o projeto de controladores denominados Robustos. Para 
tanto ser� apresentado o conceito de um controlador Robusto. A fim de modelar um
sistema sujeito a incertezas ser� apresentado alguns m�todos para que sua modelagem
matem�tica seja poss�vel. 

Para tornar o estudo mais claro ser� utilizado um sistema f�sico onde estar� sujeito a
perturba��es e/ou incertezas. Sobre este sistema ser� feito a modelagem seguindo cada um
dos processos e com estes modelos ser� efetuado uma simula��o. 

Esta simula��o ser� baseada no projeto de uma realimenta��o de estados com o intuito de
satisfazer a minimiza��o da norma $H2$ e $H_{\infty}$.

O sistema utilizado � apresentado no sistema de equa��es de estado descrito em (\ref{eq:intro_sis}).

\begin{equation}
\begin{matrix}
A=\begin{bmatrix}
0 & 1\\ 
-ba &a+b 
\end{bmatrix} &
B=\begin{bmatrix}
0\\ 
k
\end{bmatrix} 
\end{matrix}
\label{eq:intro_sis}
\end{equation}

Este sistema possui a fun��o de transfer�ncia apresentado em (\ref{eq:intro_transf}).

\begin{equation}
G(s)=\frac{k}{(s-a)(s-b)}
\label{eq:intro_transf}
\end{equation}

Os par�metros $a, b, k$ est�o sujeitos as varia��es apresentadas em (\ref{eq:intro_limit}).

\begin{equation}
\begin{matrix}
b= & -0.012725 & \\ 
k= & [k_1 \; k_2] =& [-0.4649.10^{-4} \; -0.7449.10^{-4}]\\ 
a= & [a_1 \; a_2] =& [-0.25 \; -2]
\end{matrix}
\label{eq:intro_limit}
\end{equation}

