\section{Controle Robusto}
\label{sec:robust}
%===============================================================================

Nas se��es seguintes ser� apresentado a modelagem para os tipos de incertezas.
Utilizando o sistema descrito em (\ref{eq:intro_sis}) e os limites das incertezas
descritos em (\ref{eq:intro_limit}), ser� apresentada a modelagem baseado em cada
um dos tipos apresentados a seguir.

%===============================================================================
\subsubsection{Estabilidade quadr�tica}
\label{sec:robust_quadratic}

A id�ia fundamental da estabiliza��o quadr�tica de um sistema incerto aut�nomo 
por realimenta��o linear est�tica de estados � encontrar uma lei de controle tal
que a fun��o quadr�tica definida positiva:

\begin{equation}
V(x)=x'Px com P>0
\nonumber
\end{equation}

Possua sua derivada definida negativa ao longo da trajet�ria do sistema em malha fechada:

\begin{equation}
\dot{V}(x)=x'\{(A-BK)'P+P(A-BK)\}x < 0; \forall A \in \mathbf{A}, \forall B \in \mathbf{B}
\nonumber
\end{equation}

Desta forma o que busca-se � uma matriz $P$ que satisfa�a n�o apenas uma equa��o mas
um grupo de equa��es para estabilizar o sistema de forma Robusta.

Para um sistema sem incertezas tem-se que a solu��o do problema � como abaixo:

\begin{equation}
P=W^{-1}; K=RW^{-1}
\nonumber
\end{equation}

Para o sistema abaixo com $R\equiv KP^{-1}$.

\begin{equation}
WA_0'+A_0W -B_0R-R'B_0 < 0
\nonumber
\end{equation}

%===============================================================================
\subsection{Polit�pica}
\label{sec:robust_politopica}

Para o caso de incertezas do tipo polit�pico uma condi��o necess�ria e suficiente para a
estabilidade quadr�tica do sistema incerto aut�nomo em malha fechada � que todos os v�rtices
do poliedro que constituem o conjunto de modelos possuam a mesma matriz $P$ sim�trica
definida positiva como matriz de Lyapunov.

\begin{equation}
WA_i'+A_iW-B_jR-R'B_j'< 0
\label{eq:robust_politopico}
\end{equation}

Para $\forall i=1,...,na, \forall j=1,...,nb$.

O sistema utilizado neste trabalho (\ref{eq:intro_sis}) � caracterizado utilizando a
forma polit�pica como em (\ref{eq:carac_sis_politopico}).

\begin{equation}
\begin{matrix}
A_1=\begin{bmatrix}
0 & 1\\ 
-ba_1 &a_1+b 
\end{bmatrix} \; A_2=\begin{bmatrix}
0 & 1 \\ 
-ba_2 & a_2+b 
\end{bmatrix}
\\ \; & \; \\
B_1=\begin{bmatrix}
0\\ 
k_1
\end{bmatrix}B_2=\begin{bmatrix}
0\\ 
k_2
\end{bmatrix}
\end{matrix}
\label{eq:carac_sis_politopico}
\end{equation}

%===============================================================================
\subsection{Limitada em norma}
\label{sec:carac_limit_norma}

%===============================================================================
\subsection{Diagonais}
\label{sec:carac_diagonais}


