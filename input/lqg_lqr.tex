%===============================================================================
\section{LQR e LQG em tempo discreto}
\label{sec:lqg_lqr}

%===============================================================================
\subsection{LQR para sistemas de tempo discreto}

Para o problema de LQR discreto iremos considerar o valor que o controlador prov�
um valor constante entre duas amostras de tempo ({\it{Zero-order Hold}}). E o sistema
pode ser considerado como em (\ref{eq:lqr_discrate_sys}).\cite{lq_control_dorato}

\begin{equation}
	\dot{x}=\tilde{A}x+\tilde{B}u
	\label{eq:lqr_discrate_sys}
\end{equation}

Usaremos $T$ como o per�odo de amostragem (tempo em que o valor na sa�da do controlador
� mantido constante) e $k$ � um inteiro onde $kT$ ser� o tempo em que o valor continuo
ser� amostrado.

O objetivo do controle discreto � que seja gerada uma entrada $u_k$ tal que o sistema 
em tempo continuo seja minimizado pela performance em (\ref{eq:lqr_v}).

\begin{equation}
V=\int_{t}^{t_f}(x'\tilde{Q}x+u'\tilde{R}u)d\tau+x'(t_f)Sx(t_f)
	\label{eq:lqr_v}
\end{equation}

Temos que:

\begin{equation}
x(t)= \Phi(t,kT)x_k+\int_{kT}^{t}\Phi(t,\tau)\tilde{B}u_k d\tau, \; kT \leq t \leq (k+1)T
	\label{eq:lqr_disc_x}
\end{equation}

\begin{equation}
x_{k+1}=A_k x_k+B_k u_k
	\label{eq:lqr_disc_x}
\end{equation}

Onde:
\begin{equation}
A_k=\Phi((k+1)T, kT)\nonumber
\end{equation}

\begin{equation}
B_k=\int_{kT}^{(k+1)T}\Phi((k+1), \tau)\tilde{B} d\tau \nonumber
\end{equation}

A partir de (\ref{eq:lqr_v}) temos (\ref{eq:lqr_disc}).

\begin{equation}
\int_{kT}^{(k+1)T}(x'\tilde{Q}x+u'\tilde{R}u)d\tau=x_k'Q_k x_k+2u_k'M_k x_k + u_k'R_k u_k
	\label{eq:lqr_disc}
\end{equation}

Onde:

\begin{equation}
Q_k=\int_{kT}^{(k+1)T}\Phi'(\tau, kT)\tilde{Q}\Phi(\tau, kT) \; d\tau
\nonumber
\end{equation}

\begin{equation}
M_k=\int_{kT}^{(k+1)T} H_k'(\tau)\tilde{Q}\Phi(\tau, kT) \; d\tau
\nonumber
\end{equation}

\begin{equation}
R_k=\int_{kT}^{(k+1)T} [\tilde{R}+H_k'(\tau)\tilde{Q}H_k(\tau)] \; d\tau
\nonumber
\end{equation}

\begin{equation}
H_k=\int_{kT}^{\tau}\Phi(\tau, \alpha)\tilde{B} \; d\alpha
\nonumber
\end{equation}

Podemos ent�o re escrever a performance quadr�tica dada em (\ref{eq:lqr_v}) por
(\ref{eq:lqr_v_disc}).

\begin{equation}
V(x_i,i)=\sum_{k=i}^{N-1}l(x_k, u_k, k)\; +x_N'Sx_N
\label{eq:lqr_v_disc}
\end{equation}

Onde $t_f=NT$ e:

\begin{equation}
l(x, u, k)=x'Q_k x + 2u' M_k x + u' R_k u
\nonumber
\end{equation}

%===============================================================================
\subsubsection{Otimiza��o LQR de tempo discreto}

Baseado em (\ref{eq:lqr_v_disc}) e a din�mica descrita em (\ref{eq:lqr_disc_x},
e utilizando o principio da otimalidade temos que o valor �timo para o estado $x_N$
� dado por (\ref{eq:lqr_v_optimal}).\cite{lq_control_dorato}

\begin{equation}
V^{*}(x_N,N)=x_N'Sx_N
\label{eq:lqr_v_optimal}
\end{equation}

O problema a ser minimizado � o apresentado abaixo:

\begin{equation}
\begin{matrix}
x_i'P_i x_i &= \underset{u_i}{min}[x_i'Q x_i+2u_i M x_i+u_i'R u_i \\ 
 & + (Ax_i+Bu_i)' P_{i+1}(Ax_i+Bu_i)
\end{matrix}\nonumber
\end{equation}

Dizemos que o valor de $u_i$ que coloca o sistema no ponto de valor �timo 
� $u_i^* = -K_i x_i$, onde:

\begin{equation}
K_i=(R+B'P_{i+1}A)^{-1}(B'P_{i+1}A+M)
\label{eq:lqr_k_optimal}
\end{equation}

Obtemos ent�o a solu��o da equa��o de Riccati para tempo discreto:

\begin{equation}
\begin{matrix}
P_i= & (Q+A'P_{i+1}A) \\ 
 & -(M+B'P_{i+1}A)'(R+B'P_{i+1}B)^{-1}(M+B'P_{i+1}A)
\end{matrix}
\end{equation}

Onde $P_N=S$.\cite{lq_control_dorato}


%===============================================================================
%\subsection{Problema de filtragem �tima no caso discreto}

%===============================================================================
