%===============================================================================
\section{Minimizac�o da norma H2}
\label{sec:normH2}

Considerando o sistema apresentado em (\ref{eq:robust_sis_incert}) em malha fechada
com $\Delta \equiv 0$ e condic�o inicial nula. Um crit�rio normalmente utilizado 
� a norma $H_2$ da func�o de transfer�ncia entre a entrada das perturbac�es $\omega$
e a sa�da $z$. A norma $H_2$ pode ser calculada como em (\ref{eq:normh2_def}).

\begin{equation}
\left \| T(s) \right \|_2 \equiv \gamma _2=\sqrt{Tr(B_{\omega}'P_o B_{\omega})}
\label{eq:normh2_def}
\end{equation}

Onde :

\begin{equation}
P_o=\int_{0}^{\infty}((G-HK))e^{(A_0-BK)}B_{\omega})'((G-HK))e^{(A_0-BK)}B_{\omega})dt
\nonumber
\end{equation}

{\it{Interpretac�o estoc�stica da norma $H_2$}}: Se considerarmos $\omega(t)$ como sendo 
ruido branco, ent�o a norma $H_2$ de $T(s)$ � o valor da vari�ncia assint�tica da sa�da $z(t)$:

\begin{equation}
\left \| T(s) \right \|_2=\sqrt{\lim_{t\rightarrow \infty}E(z(t)'z(t))}
\nonumber
\end{equation}

{\it{Interpretac�o deterministica da norma $H_2$}}: D� a id�ia da energia da sa�da $z(t)$ 
em resposta as condic�es iniciais nulas.

\begin{equation}
\int_{0}^{\infty}z(t)'z(t)dt=x_0'P_o x_0
\nonumber
\end{equation}

Assim tendo que a norma pode ser apresentada como a seguir:

\begin{equation}
\left \| T(s) \right \|_2=\sqrt{\sum_{i=1}^{n_w} \int_{0}^{\infty}\left \| z^{i}(t) \right \|^2 dt}
\nonumber
\end{equation}
